\documentclass[10pt]{article}

\usepackage[english]{babel}
\usepackage[utf8x]{inputenc}
\usepackage{amsmath}
\usepackage{graphicx}
\usepackage[colorinlistoftodos]{todonotes}
\usepackage{listings}
\usepackage{glossaries}
\usepackage{placeins}
\usepackage{fixltx2e}
\usepackage{scrpage2}
\usepackage{scrtime}
\clearscrheadfoot
\pagestyle{scrheadings}
\usepackage[
top    = 2.5cm,
bottom = 3cm,
left   = 3cm,
right  = 3cm]{geometry}
\setcounter{secnumdepth}{4}
\title{3rd Sprint}

\author{RoboNav}
\date{\today}


\begin{document}


\begin{titlepage}
\begin{center}
% Oberer Teil der Titelseite:

%NEEDS TO BE OUT COMMENTED sometimes
%\includegraphics[width=1.0\textwidth]{../../pictures/robonavlogo}\\  
\includegraphics[width=0.5\textwidth]{pictures/logo}\\  

\LARGE TGM - HTBLuVA Wien XX \\ Informationstechnologie  \\[1.5cm]

% Title
\rule{14cm}{1mm}
{ \huge \bfseries \\[0.4cm]  \huge INSY \\ \LARGE Backup mit Mysql und Postgresql \\[0.4cm] }

\rule{14cm}{1mm}



% Author and supervisor
\noindent 
\vspace{7cm}
\small
\begin{center}
  \begin{tabular}{ | p{0.1\textwidth} | p{0.2\textwidth} | p{0.2\textwidth} | p{0.1\textwidth} | p{0.2\textwidth} |}
    \hline
\textbf{Version} & \textbf{Autor} & \textbf{Datum} & \textbf{Status} & \textbf{Kommentar} \\ 
    \hline 
    \hline
0.1 & Siegel & 2014.11.28 & Draft &  Erstellung Dokument   \\ 
    \hline
  \end{tabular}
\end{center}

\vfill

% Bottom of the page
{\small Version of this document: \today ~at  \thistime    }
\end{center}

%\end{center}
\end{titlepage}


%HEADER AND FOOTER
\pagenumbering{arabic}
\ohead{\headmark}
\ifoot{© Haidn, Siegel}
\ofoot{\pagemark /\pageref{LastPage}}
\tableofcontents
\section{Aufgabenstellung}
Untersuchen Sie die Backup-Tools von MySQL (mysqldump, mysqlhotcopy, ibbackup) und PostgreSQL (pg\_dump) und lösen Sie folgende Aufgaben:
\\ \\
Finden und dokumentieren Sie (für ihr System OS/DBMS) die etsprechenden Optionen der Tools für folgende Anforderungen: \\
Speichern einer/mehrerer/aller Datenbanken des Systems in einer Datei mit/ohne Datenbankstruktur, Trigger und Stored-Routines 
Verwendung der "IF EXISTS"- und "DROP"-Klausel unter MySQL bzw. PostgreSQL \\
Logisches vs. Physisches Backup: Was sind die Vor- bzw. Nachteile der beiden Arten und worauf muss man achten
Online-Backup: Wie kann man einen Dump der DB während des Betriebs ausführen (Locking, ...)
Wie können Sie auf gemieteten DB-Servern (remote) ebenfalls Backups ausführen? Geben Sie zwei Möglichkeiten an.
\\ \\
Wie könnte man die Backupvarianten aus Punkt 1 automatisieren (Uhrzeit als Trigger)? Geben Sie entsprechend für ihr Betriebssystem (Windows, Linux, Mac, ...) Möglichkeiten an.
Verwendung eines Zeitstempels zur Speicherung der Dumps (in den Filenamen inkludiert; z.B. DBNAME\_20100413\_0952.sql)
\\ \\
Abgaberichtlinien:
 \\ \\
PDF-Dokument, ca. 15 Seiten, formatiert und strukturiert ähnlich wie das Technik-/Machbarkeits-Kapitel der Diplomarbeit (Zitate, Quellen, Fußnoten, Tabellen, Grafiken, Screenshots, Inhaltsverzeichnis, ...)
Bitte in Zweier-Teams arbeiten, alle im Team mitarbeitenden Autoren müssen aber in der Lage sein, jedes Thema/Detail auch selbst zu präsentieren. \\
Arbeitsaufwand ca. 10 Stunden pro Team.  \\
Präsentation am 9.12.2014

\section{Arbeitszeit}
\begin{table}[h]

\begin{tabular}{|p{0.4\textwidth}|p{0.2\textwidth}|p{0.2\textwidth}|}
\hline
\textbf{Task}    & \textbf{Person}                                               & \textbf{Time in hours                              } \\ \hline \hline
Aufsetzten und ueberpfruefen der Datenbanken & \begin{tabular}[c]{c} Haidn \\Siegel\end{tabular} & \begin{tabular}[c]{c}1\\ 1\end{tabular}    \\ \hline 

Task2 & \begin{tabular}[c]{c}Haidn\\ Siegel\end{tabular} & \begin{tabular}[c]{c}1\\ 1\end{tabular}    \\ 

\hline \hline
Total & \begin{tabular}[c]{c}Haidn\\ Siegel\end{tabular} & \begin{tabular}[c]{c}2\\2\end{tabular}   \\ \hline 
\textbf{Total Team} & & \textbf{XXX hours}  \\ \hline 
\end{tabular}
\end{table}
\section{Ausgangssituationen}
\subsection{Hannah}
Betriebsystem: Windows 8.1 \\
Datenbank: Mysql Version 14.14, Distribution 5.5.40 \\
Datenbank laeuft auf VM: Ubuntu 14.04 LTS mit der IP 192.168.117.131
\subsection{Martin}

\section{Theorie Backup}
\subsection{Allgemein} \todo{Section weg sondern als text unter Section}
Vorkerungen um im Bedarfsfall eine Wiederherstellung durchfuehren zu koennen -> Datensicherungskonzept
\subsection{Die 5 W Fragen} \todo{Bessere bennenung.}
\subsubsection{Warum braucht man Backups?}
Backups werden benoetigt, um nach einem Ausfall eine Wiederherstellung (Recovery) durchfuehren zu koennen.
Somit kann ein Datenverlust vermieden werden, was besonders wichtig ist, denn Daten gelten in der heutigen Zeit als eines der Wichtigsten Assets eines Unternehmens.
\subsubsection{Wofuer braucht man Backups?}
\textbf{Datensicherung} \\


\textbf{Archivierung}	\\
\subsubsection{Was wird gesichtert? - Logisches Backup vs. Physiches Backup}
\textbf{Logisches Backup} \\
(DDL - Script) + Inserts
\textbf{Physiches Backup} \\
(01010101110111...) 
\textbf{Vergleich} \\
Logische Datensicherung braucht mehr platz
Insert into values ‘2014-11-26’ = (20 + x) * 2 Bytes
TIMESTAMP - LONG - 8 Byte 
Vorteil der Logischen: Struktur vorhanden, leichter verwendbar

\subsubsection{Wie wird gesichtert?}
\textbf{Komplettsicherung} 
Alle Daten jedes mal 
 \textbf{Differenzsicherung}
 nur die Aenderungen werden abgespeichert

\textbf{Inkrementelle Sicherung} 
- Alles und alle Aenderungen
(Aenderungen 1, 2, 3, … werden abgespeichert und muessen in der richtigen Reihenfolge eingespeichert werden)
\subsubsection{Wann wird gesichtert?}

\textbf{Online 'hot'} \\
\textbf{Offline 'cold'} \\
\textbf{Vergleich} \\
Online ‘hot’ : komplitzierter (je nach einzel oder mehrbenutzer Betrieb)
Offline ‘cold’ :  schneller, sicherer





\subsection{Online Backup}

\section{Speichern der Datenbanken in eine Datei}
\subsection{Mysql}



---------------------------------
	


backup mysqldump -u root -p [database\_name] > dumpfilename.sql
password:

restore mysql -u root -p [database\_name] < dumpfilename.sql
password:
\subsubsection{Speichern einer Datenbank in eine Datei mit Datenbankstruktur}
\subsubsection{Speichern einer Datenbank in eine Datei ohne Datenbankstruktur}
\subsubsection{Speichern mehrerer Datenbanken in eine Datei mit Datenbankstruktur}
\subsubsection{Speichern mehrerer Datenbanken in eine Datei ohne Datenbankstruktur}
\subsubsection{Speichern aller Datenbanken in eine Datei mit Datenbankstruktur}
\subsubsection{Speichern aller Datenbanken in eine Datei ohne Datenbankstruktur}
\subsubsection{Speichern von Triggern / Stored Routines}
\subsubsection{Drop Klauseln}

\subsection{Psql}
\subsubsection{Speichern einer Datenbank in eine Datei mit Datenbankstruktur}
\subsubsection{Speichern einer Datenbank in eine Datei ohne Datenbankstruktur}
\subsubsection{Speichern mehrerer Datenbanken in eine Datei mit Datenbankstruktur}
\subsubsection{Speichern mehrerer Datenbanken in eine Datei ohne Datenbankstruktur}
\subsubsection{Speichern aller Datenbanken in eine Datei mit Datenbankstruktur}
\subsubsection{Speichern aller Datenbanken in eine Datei ohne Datenbankstruktur}
\subsubsection{Speichern von Triggern / Stored Routines}
\subsubsection{Drop Klauseln}
\section{Online Backup}
\subsection{Mysql}
\subsection{Psql}

\section{Remote Backups}
% 2 Moeglichkeiten!

\section{Automatisierung von Backups}
\subsection{Mysql}
\subsubsection{Uhrzeit als Trigger}
\subsubsection{Verwendung eines Zeitstempels zur Speicherung des Dumps}
\subsection{Psql}
\subsubsection{Uhrzeit als Trigger}
\subsubsection{Verwendung eines Zeitstempels zur Speicherung des Dumps}


\section{Problems}


\subsection{Mysql connection didn't work out}
When tring to connect to the mysql Database, the following error was thrown:
\texttt{ERROR 2002 (HY000): Can't connect to local MySQL server through socket '/var/run/mysqld/mysqld.sock'} \\
The following steps had to be done: \\
\\ 
\begin{enumerate}
\item Super user or sudo should be used. (\texttt{sudo su}) 
\item The file \texttt{/etc/mysql/my.cnf} must be opened 
\item The bind address had to be changed to \texttt{127.0.0.1}
\item Then \texttt{service mysql restart} must be executed
\end{enumerate}
\cite{mysqlProblem1}



\subsection{Easy Bibliography}
\begin{thebibliography}{56}

  
 \bibitem{mysqlProblem1} 
  \textbf{ERROR 2002: Can't connect to local MySQL server through socket \\ StackOverflow} \\ Peter Mortensen Apr 13 at 10:45 and  rshahriar Oct 11 '12 at 6:27 \\
  \textit{http://stackoverflow.com/questions/11657829/error-2002-hy000-cant-connect-to-local-mysql-server-through-socket-var-run}
  \newline last used: 2014.12.05, 13:15
 
 \bibitem{mysqldumpman} 
  \textbf{Mysql Manual 5.5}, mysqldump — A Database Backup Program\\
  \textit{http://dev.mysql.com/doc/refman/5.5/en/mysqldump.html}
  \newline last used: 2014.12.05, 13:55
  
   \bibitem{mysqldumpman} 
  \textbf{Mysql Manual 5.5}, mysqldump — A Database Backup Program\\
  \textit{http://dev.mysql.com/doc/refman/5.5/en/mysqldump.html}
  \newline last used: 2014.12.05, 13:55
  



\end{thebibliography}
\end{document}
