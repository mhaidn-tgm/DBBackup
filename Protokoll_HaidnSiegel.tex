\documentclass[10pt]{article}

\usepackage[english]{babel}
\usepackage[utf8x]{inputenc}
\usepackage{amsmath}
\usepackage{graphicx}
\usepackage[colorinlistoftodos]{todonotes}
\usepackage{listings}
\usepackage{glossaries}
\usepackage{placeins}
\usepackage{fixltx2e}
\usepackage{scrpage2}
\usepackage{scrtime}
\clearscrheadfoot
\pagestyle{scrheadings}
\usepackage[
top    = 2.5cm,
bottom = 3cm,
left   = 3cm,
right  = 3cm]{geometry}
\setcounter{secnumdepth}{4}
\definecolor{dkgreen}{rgb}{0,0.6,0}
\definecolor{gray}{rgb}{0.5,0.5,0.5}
\definecolor{mauve}{rgb}{0.58,0,0.82}

\newcommand{\executeiffilenewer}[3]{%
\ifnum\pdfstrcmp{\pdffilemoddate{#1}}%
{\pdffilemoddate{#2}}>0%
{\immediate\write18{#3}}\fi%
}
\newcommand{\includesvg}[1]{%
\executeiffilenewer{#1.svg}{#1.pdf}%
{inkscape -z -D --file=#1.pdf --export-pdf=#1.pdf --export-latex}%
\input{#1.pdf_tex}%
}

\lstset{frame=tb,
  language=Java,
  aboveskip=3mm,
  belowskip=3mm,
  showstringspaces=false,
  columns=flexible,
  basicstyle={\small\ttfamily},
  numbers=none,
  numberstyle=\tiny\color{gray},
  keywordstyle=\color{blue},
  commentstyle=\color{dkgreen},
  stringstyle=\color{mauve},
  breaklines=true,
  breakatwhitespace=true
  tabsize=3
}


\begin{document}


\begin{titlepage}
\begin{center}
% Oberer Teil der Titelseite:

%NEEDS TO BE OUT COMMENTED sometimes
%\includegraphics[width=1.0\textwidth]{../../pictures/robonavlogo}\\  
\includegraphics[width=0.5\textwidth]{pictures/logo}\\  

\LARGE TGM - HTBLuVA Wien XX \\ Informationstechnologie  \\[1.5cm]

% Title
\rule{14cm}{1mm}
{ \huge \bfseries \\[0.4cm]  \huge INSY \\ \LARGE Backup mit Mysql und Postgresql \\[0.4cm] }

\rule{14cm}{1mm}



% Author and supervisor
\noindent 
\vspace{7cm}
\small
\begin{center}
  \begin{tabular}{ | p{0.1\textwidth} | p{0.2\textwidth} | p{0.2\textwidth} | p{0.1\textwidth} | p{0.2\textwidth} |}
    \hline
\textbf{Version} & \textbf{Autor} & \textbf{Datum} & \textbf{Status} & \textbf{Kommentar} \\ 
    \hline 
    \hline
0.1 & Siegel & 2014.11.28 & Draft &  Erstellung Dokument   \\ 
    \hline
  \end{tabular}
\end{center}

\vfill

% Bottom of the page
{\small Version of this document: \today ~at  \thistime    }
\end{center}

%\end{center}
\end{titlepage}


%HEADER AND FOOTER
\pagenumbering{arabic}
\ohead{\headmark}
\ifoot{© Haidn, Siegel}
\ofoot{\pagemark}
\tableofcontents
\section{Task description}
Untersuchen Sie die Backup-Tools von MySQL (mysqldump, mysqlhotcopy, ibbackup) und PostgreSQL (pg\_dump) und lösen Sie folgende Aufgaben:
\\ \\
Finden und dokumentieren Sie (für ihr System OS/DBMS) die etsprechenden Optionen der Tools für folgende Anforderungen: \\
Speichern einer/mehrerer/aller Datenbanken des Systems in einer Datei mit/ohne Datenbankstruktur, Trigger und Stored-Routines 
Verwendung der "IF EXISTS"- und "DROP"-Klausel unter MySQL bzw. PostgreSQL \\
Logisches vs. Physisches Backup: Was sind die Vor- bzw. Nachteile der beiden Arten und worauf muss man achten
Online-Backup: Wie kann man einen Dump der DB während des Betriebs ausführen (Locking, ...)
Wie können Sie auf gemieteten DB-Servern (remote) ebenfalls Backups ausführen? Geben Sie zwei Möglichkeiten an.
\\ \\
Wie könnte man die Backupvarianten aus Punkt 1 automatisieren (Uhrzeit als Trigger)? Geben Sie entsprechend für ihr Betriebssystem (Windows, Linux, Mac, ...) Möglichkeiten an.
Verwendung eines Zeitstempels zur Speicherung der Dumps (in den Filenamen inkludiert; z.B. DBNAME\_20100413\_0952.sql)
\\ \\
Abgaberichtlinien:
 \\ \\
PDF-Dokument, ca. 15 Seiten, formatiert und strukturiert ähnlich wie das Technik-/Machbarkeits-Kapitel der Diplomarbeit (Zitate, Quellen, Fußnoten, Tabellen, Grafiken, Screenshots, Inhaltsverzeichnis, ...)
Bitte in Zweier-Teams arbeiten, alle im Team mitarbeitenden Autoren müssen aber in der Lage sein, jedes Thema/Detail auch selbst zu präsentieren. \\
Arbeitsaufwand ca. 10 Stunden pro Team.  \\
Präsentation am 9.12.2014

\section{Working time}
\subsection{Estimated}
\begin{table}[h]

\begin{tabular}{|p{0.5\textwidth}|p{0.2\textwidth}|p{0.2\textwidth}|}
\hline
\textbf{Task}    & \textbf{Person}                                               & \textbf{Time in hours                              } \\ \hline \hline
Setting up the Databases & \begin{tabular}[c]{c} Haidn \\Siegel\end{tabular} & \begin{tabular}[c]{c}1\\ 1\end{tabular}    \\ \hline 
Getting some informations about backups & \begin{tabular}[c]{c} Haidn \\Siegel\end{tabular} & \begin{tabular}[c]{c}2\\ 1\end{tabular}    \\ \hline 
mysqldump & \begin{tabular}[c]{c} Haidn \\Siegel\end{tabular} & \begin{tabular}[c]{c}2\\ 2\end{tabular}    \\ \hline 
mysqlbackup & \begin{tabular}[c]{c} Haidn \\Siegel\end{tabular} & \begin{tabular}[c]{c}0\\ 2\end{tabular}    \\ \hline 
system-level commands & \begin{tabular}[c]{c} Haidn \\Siegel\end{tabular} & \begin{tabular}[c]{c}2\\ 0\end{tabular}    \\ \hline 
mysqlhotcopy & \begin{tabular}[c]{c} Haidn \\Siegel\end{tabular} & \begin{tabular}[c]{c}1.5\\ 1.5\end{tabular}    \\ \hline 

\hline \hline
Total & \begin{tabular}[c]{c}Haidn\\ Siegel\end{tabular} & \begin{tabular}[c]{c}2\\2\end{tabular}   \\ \hline 
\textbf{Total Team} & & \textbf{XXX hours}  \\ \hline 
\end{tabular}
\end{table}
\subsection{End}
\begin{table}[h]
\begin{tabular}{|p{0.4\textwidth}|p{0.2\textwidth}|p{0.2\textwidth}|}
\hline
\textbf{Task}    & \textbf{Person}                                               & \textbf{Time in hours                              } \\ \hline \hline
Setting up the Databases & \begin{tabular}[c]{c} Haidn \\Siegel\end{tabular} & \begin{tabular}[c]{c}0.5\\ 2\end{tabular}    \\ \hline 


\hline \hline
Total & \begin{tabular}[c]{c}Haidn\\ Siegel\end{tabular} & \begin{tabular}[c]{c}2\\2\end{tabular}   \\ \hline 
\textbf{Total Team} & & \textbf{XXX hours}  \\ \hline 
\end{tabular}
\end{table}
\FloatBarrier
\section{Working situation}
\subsection{Hannah}
OS: Windows 8.1 \\
Database: Mysql Version 14.14, Distribution 5.5.40 \\
VM on which the DB runs: Ubuntu 14.04 LTS , it's IP: 192.168.117.131
\subsection{Martin}
\todo{martin}
\newpage
\section{Backups general}
\subsection{Why should backups be done?}
After a drop-out, a recovery must me done. \\
To avoid a loss of data, a backup must be done before an drop-out occurs and it always should be current. \\ \\
Backups are not only needed for recovery purposes, but also for archival storage purposes.
\subsection{Logical versus physical backups}
Logical backups save information represented as logical database structure (e.g. \texttt{Create Database}, \texttt{Create Table}  statements) and content (e.g \texttt{Insert} statements). Physical backups consist of raw copies of the directories and files that store database contents. \\ \\
Logical backups :
\begin{enumerate}
\item Backup is done by querying the MySQL server
\item Slower than physical methods
\item Output is larger than physical methods
\item The Backup and the Restore can be done either for all databases (server level), only for one database (database level) or only for specific tables (table level)
\item Doesent include log or config files
\item Backups are mostly done with the database still running
\item Can be easily imported
\item Backups stored in logical format are machine independent!
\end{enumerate}\cite{mysqlbackupandrectypesman}
\\ \\Physical backups :
\begin{enumerate}
\item Exact copy of database files which are stored on the disk
\item Output is more compact than logical ones
\item The granularity of the data that can be stored depends on the engine (e.g. InnoDb shares files with other tables..) 
\item Can include log or config files
\item Backups stored in logical format are machine dependent!
\item Backups are seldom done with the database running, and if then the database files must be locked.
\item Can be easily imported
\end{enumerate} \cite{mysqlbackupandrectypesman}
\subsection{Full versus incremental backups}
"Some file system implementations enable “snapshots” to be taken. These provide logical copies of the file system at a given point in time, without requiring a physical copy of the entire file system. [..] MySQL itself does not provide the capability for taking file system snapshots. It is available through third-party solutions such as Veritas, LVM, or ZFS.", \cite{mysqlbackupandrectypesman}
\\ \
"A full backup includes all data managed by a MySQL server at a given point in time. An incremental backup consists of the changes made to the data during a given time span (from one point in time to another). MySQL has different ways to perform full backups, such as those described earlier in this section. Incremental backups are made possible by enabling the server's binary log, which the server uses to record data changes. \\
Incremental recovery is recovery of changes made during a given time span. This is also called point-in-time recovery because it makes a server's state current up to a given time. Point-in-time recovery is based on the binary log and typically follows a full recovery from the backup files that restores the server to its state when the backup was made. Then the data changes written in the binary log files are applied as incremental recovery to redo data modifications and bring the server up to the desired point in time.", \cite{mysqlbackupandrectypesman}


%\textbf{Komplettsicherung} 
%Alle Daten jedes mal 
% \textbf{Differenzsicherung}
% nur die Aenderungen werden abgespeichert
%\textbf{Inkrementelle Sicherung} 
%- Alles und alle Aenderungen
%(Aenderungen 1, 2, 3, … werden abgespeichert und muessen in der richtigen Reihenfolge eingespeichert werden)
\subsubsection{Online versus Offline backups}
Online backups, also called hot backups, take place while the server is running so that the database can still be used.\\
Offline backups, also called cold backups, take place while the server is not running and therefore the database is not availiable. \\ \\
Online backups :
\begin{enumerate}
\item Clients can still access the database
\item The backup must be made carefully, in order to secure, that the clients have not changed informations in the mean time that could compromise the backup's integrity
\end{enumerate} \cite{mysqlbackupandrectypesman}\\ \\
Offline backups :
\begin{enumerate}
\item Clients can not access the database
\item The backup is easier
\end{enumerate}
A similar distinction between online and offline applies for recovery operations.
 \cite{mysqlbackupandrectypesman}
\newpage
\section{Mysql Backup}
\subsection{Logical Backup into Files with mysqldump}
Mysql has an option called mysqldump. Mysql dump can connect to local or remote servers.\cite{mysqlbackupandrectypesman}
\subsubsection{Performing a database dump}
To do a backup from only one database, the following command needs to be executed: \\
\texttt{ mysqldump -u root -p [database\_name] \textgreater ~dumpfilename.sql } \\
After typing in the password, a file will be available which, in this case is called \texttt{dumpfilename.sql}.\\ 
The content of the dumpfile is the following:
\begin{lstlisting}    
-- MySQL dump 10.13  Distrib 5.5.40, for debian-linux-gnu (x86_64)
--
-- Host: localhost    Database: insy1
-- ------------------------------------------------------
-- Server version	5.5.40-0ubuntu0.14.04.1

/*!40101 SET @OLD_CHARACTER_SET_CLIENT=@@CHARACTER_SET_CLIENT */;
/*!40101 SET @OLD_CHARACTER_SET_RESULTS=@@CHARACTER_SET_RESULTS */;
/*!40101 SET @OLD_COLLATION_CONNECTION=@@COLLATION_CONNECTION */;
/*!40101 SET NAMES utf8 */;
/*!40103 SET @OLD_TIME_ZONE=@@TIME_ZONE */;
/*!40103 SET TIME_ZONE='+00:00' */;
/*!40014 SET @OLD_UNIQUE_CHECKS=@@UNIQUE_CHECKS, UNIQUE_CHECKS=0 */;
/*!40014 SET @OLD_FOREIGN_KEY_CHECKS=@@FOREIGN_KEY_CHECKS, FOREIGN_KEY_CHECKS=0 */;
/*!40101 SET @OLD_SQL_MODE=@@SQL_MODE, SQL_MODE='NO_AUTO_VALUE_ON_ZERO' */;
/*!40111 SET @OLD_SQL_NOTES=@@SQL_NOTES, SQL_NOTES=0 */;

--
-- Table structure for table `Abteilung`
--

DROP TABLE IF EXISTS `Abteilung`;
/*!40101 SET @saved_cs_client     = @@character_set_client */;
/*!40101 SET character_set_client = utf8 */;
CREATE TABLE `Abteilung` (
  `aname` varchar(255) NOT NULL,
  `sync_state` enum('current','old','new','syncing','deleting') NOT NULL DEFAULT 'new',
  PRIMARY KEY (`aname`,`sync_state`)
) ENGINE=InnoDB DEFAULT CHARSET=latin1;
/*!40101 SET character_set_client = @saved_cs_client */;
...
\end{lstlisting}    
Even trough that in this case there is only one create table command, there is a lot of bulk. \\
In the next example, the Inserts into a table can be seen, and here it actually is not too much bulk.
Still, whenever saving a database dump like this, the file will be quite big, which might be a big disadvantage! 
\begin{lstlisting}    
INSERT INTO `Person` VALUES 
('Aly','Ahmed','Facility Management','Doppelte gasse','current'),
('Dominik','Scholz','IT','Schwarze gasse','current'),
('Elias','Frantar','Kindergarten','Heiligenstadt gasse','current'),
('Hannah','Siegel','HR','Max Kahrer gasse','current'),
('Jakob','Saxinger','Kueche','Max Soundso gasse','current'),
('Martin','Haidn','Managment','Gruene gasse','current'),...
\end{lstlisting}    
\textbf{Backup of only one table using mysqldump}\\
\texttt{ mysqldump -u root -p [database\_name] [table\_name] \textgreater ~dumpfilename.sql } \\ \\
\textbf{Backup of more than one database using mysqldump}\\
\texttt{ mysqldump -u root -p --databases [database\_name1] [database\_name2] \textgreater ~dumpfilename.sql } \\ 
\textbf{Backup of all databases using mysqldump}\\
\texttt{ mysqldump -u root -p --all-databases \textgreater ~dumpfilename.sql } \\ \\
\textbf{Backup of only the structure without any data}\\
\texttt{mysqldump -u root -p [-d|--no-data] [database\_name] \textgreater ~dumpfilename.sql }
\subsubsection{Drop statements}
If drop-statements should be added, the following parameters can simply be added:
\begin{table}[h]
\begin{tabular}{ll}
\textbf{Format}     & \textbf{Description}                                                \\
--add-drop-database & Adds a DROP DATABASE statement before each CREATE DATABASE statement \\
--add-drop-table    & Adds a DROP TABLE statement before each CREATE TABLE statement       \\
\end{tabular}
\end{table} \cite{mysqldumpman}
The option \texttt{--add-drop-trigger} was supported in version 5.1, but it is not availiable anymore in 5.5:  
\begin{lstlisting}    
root@ubuntu:/var/lib/mysql# mysqldump -u root -p insy2 --add-drop-trigger > df1.sql
mysqldump: unknown option '--add-drop-trigger' 
\end{lstlisting}    
\subsubsection{Performing a data restore}
After having dropped the database, the following commad was restoring the data. this has worked out fine:
To do a backup from only one database, the following command needs to be executed: \\
\texttt{mysql -u root -p [database\_name] \textless dumpfilename.sql} \\ \\
With this type of recovery, even the triggers have been imported.
\newpage
\subsection{Physical backup with mysqlbackup}
Mysql is only available for the mysql Enterprise Edition.
\todo{more information}
\begin{figure}[!h]
	\begin{center}
		\includegraphics[width=0.7\linewidth]{pictures/mysqlentbackup}
		\caption{Difference between mysqlbackup and mysqldump \cite{mysqlenterprisebackup}}
		\label{differenceent}
	\end{center}
\end{figure}\FloatBarrier
\subsection{Restoring from another Database}
Also, a database could be copied from another using these commands.
We have tried this out an it worked.
\begin{lstlisting}    
MyISAM:
CREATE TABLE db2.mytable LIKE db1.mytable;
ALTER TABLE db2.mytable DISABLE KEYS;
INSERT INTO db2.mytable SELECT * FROM db1.mytable;
ALTER TABLE db2.mytable ENABLE KEYS;
 \end{lstlisting}    
 \begin{lstlisting}    
INNODB:
CREATE TABLE db2.mytable LIKE db1.mytable;
INSERT INTO db2.mytable SELECT * FROM db1.mytable;
  \end{lstlisting}    
\cite{so1}
\subsection{Physical backup using File system commands}
\label{sec:cpmysql}
In Order to copy the files, the location must be found out. This can be done with the \texttt{select @@datadir} command within mysql.
\begin{figure}[!h]
	\begin{center}
		\includegraphics[width=0.4\linewidth]{pictures/datadir_mysql}
		\caption{Output of the select @@datadir command}
		\label{differenceent}
	\end{center}
\end{figure}
\FloatBarrier
In figure \ref{content}, the content of the insy1 database can be seen. The only MyISAM table is 'Logged'.\\
Also, each trigger has its own file.
\FloatBarrier
\begin{figure}[!h]
	\begin{center}
		\includegraphics[width=0.8\linewidth]{pictures/ls_mysql_insy1}
		\caption{Content of the ver/etc/mysql folder}
		\label{content}
	\end{center}
\end{figure}
\FloatBarrier
\subsubsection{MyISAM}
Copying files when using a MyISAM Database, is possible, because every Table maps to exactly one file. \\
"However, you cannot just move the .frm. You must move all components.", \cite{so1} \\
There are three files (see also in figure \ref{content}) which have something to do with the table:
\begin{itemize}
\item /var/lib/mysql/insy1/Logged.frm
\item /var/lib/mysql/insy1/Logged.MYD (Table Database)
\item /var/lib/mysql/insy1/Logged.MYI (Table Indexes)
\end{itemize}
When simply coping these files, dropping the table and coping them back, \\ and then calling a \texttt{Select * from Logged} command, the following error message occurs:
\begin{lstlisting}
 ERROR 1017 (HY000): Can't find file: './insy1/Logged.frm' (errno: 13)
\end{lstlisting}
When searching for a solution, there are some, but mostly it says that it is really better backing mysql databases up with \texttt{mysqlhotcopy}.
\subsubsection{InnoDB}
On the other hand, doing a backup when coping files, is "risky (near suicidal) with InnoDB.",\cite{so1}
Therefore, whenever using InnoDB, a backup should be done with \texttt{mysqldump}.
\subsection{mysqlhotcopy}
"mysqlhotcopy is a Perl script that was originally written and contributed by Tim Bunce. It uses FLUSH TABLES, LOCK TABLES, and cp or scp to make a database backup. It is a fast way to make a backup of the database or single tables, but it can be run only on the same machine where the database directories are located. mysqlhotcopy works only for backing up MyISAM and ARCHIVE tables. It runs on Unix.",\cite{mysqlhotcopyman}
\subsubsection{Copying the files}
Using the mysqlhotcopy is a really easy way to copy the files. In the following example, it can be seen, what command must be used and the output. \\
\begin{lstlisting}
hsiegel@ubuntu:~/Documents/sync_het_db/Create$ sudo /usr/bin/mysqlhotcopy -u root -p secret_password insy1 /home/hsiegel/Desktop --allowold --keepold
Flushed 5 tables with read lock (`insy1`.`Abteilung`, `insy1`.`Logged`, `insy1`.`Person`, `insy1`.`Teilnehmer`, `insy1`.`Veranstaltung`) in 0 seconds.
Locked 0 views () in 0 seconds.
Copying 24 files...
Copying indices for 0 files...
Unlocked tables.
mysqlhotcopy copied 5 tables (24 files) in 0 seconds (0 seconds overall).
\end{lstlisting}
This simply copies the files from /var/lib/mysql/insy1 to the source destination.
\begin{figure}[!h]
	\begin{center}
		\includegraphics[width=0.8\linewidth]{pictures/lsdesktop}
		\caption{Content of the insy1 folder, which has been generated using mysqlhotcopy}
		\label{content}
	\end{center}
\end{figure}

\subsubsection{Restoring}
"To restore the backup from the mysqlhotcopy backup, simply copy the files from the backup directory to the /var/lib/mysql/{db-name} directory. Just to be on the safe-side, make sure to stop the mysql before you restore (copy) the files. After you copy the files to the /var/lib/mysql/{db-name} start the mysql again.",\cite{mysqlhtcoptut}
In our example, restoring after having done a mysqlhotcopy didn't work either, with the same error message as before:
\begin{lstlisting}
 ERROR 1017 (HY000): Can't find file: './insy1/Logged.frm' (errno: 13)
\end{lstlisting}
\subsection{Backup of Triggers / Stored Routines}
When using mysqldump, the triggers have been saved automatically and the import has not been a problem. \\
Because the mysqlhotcopy didn't work, we couldn't evaluate it by our selves if the triggers were restored. But normally, it should be possible using mysqlhotcopy as well.

\subsection{Online Backup}
With MySQL Enterprise Backup, a hot backup is possible. \\
Also, mysqlhotcopy, as the name already says, is also able to perform hot backups, because the Perl script handles the locking.
Mysqldump is executing Select statements on the database, therefore it is a hot copy as well, but in this case, it might be really slow for users and the task itself.
\subsection{Remote Backups}
\subsubsection{Using mysqldump to perform remote backups}
You can specify the server name as an option to mysqldump: 
\texttt{mysqldump --host [server\_name] [database\_name] \textgreater ~dumpfilename.sql}
\subsubsection{Using ftp to perform remote backups}
As described in section \ref{sec:cpmysql}, the files can simply be copied, and this process of copying can also be performed using ftp.
\subsection{Automised Backups}
An important way of performing backups is that they can also be automatised. This has the advantage, that the administrator doesn't has to think about doing a backup and therefore is not able to forget it, and also backups can then be scheduled at times, when the database server might not be very busy, so the backup is faster and the client do not notice that a backup task has just been performed. 


\todo{Autobackup - needs to be tested::}
https://www.digitalocean.com/community/tutorials/how-to-backup-mysql-databases-on-an-ubuntu-vps
http://serverzeit.de/tutorials/backup/automysqlbackup
http://www.bjoerne.com/mysql-backups-mit-automysqlbackup-erstellen/
http://www.debianhelp.co.uk/mysqlscript.htm

\subsubsection{Uhrzeit als Trigger}
\subsubsection{Verwendung eines Zeitstempels zur Speicherung des Dumps}

% Bestes mysql backup tool.

\todo{INKREMENTELLE BACKUPS MYSQL?????}
\newpage
\section{Problems}

\subsection{Mysql connection didn't work out}
When trying to connect to the mysql Database, the following error was thrown:
\texttt{ERROR 2002 (HY000): Can't connect to local MySQL server through socket '/var/run/mysqld/mysqld.sock'} \\
The following steps had to be done: \\
\\ 
\begin{enumerate}
\item Super user or sudo should be used. (\texttt{sudo su}) 
\item The file \texttt{/etc/mysql/my.cnf} must be opened 
\item The bind address had to be changed to \texttt{127.0.0.1}
\item Then \texttt{service mysql restart} must be executed
\end{enumerate}
\cite{mysqlProblem1}


\newpage
\begin{thebibliography}{56}

  
 \bibitem{mysqlProblem1} 
  \textbf{ERROR 2002: Can't connect to local MySQL server through socket \\ StackOverflow} \\ Peter Mortensen Apr 13 at 10:45 and  rshahriar Oct 11 '12 at 6:27 \\
  \textit{http://stackoverflow.com/questions/11657829/error-2002-hy000-cant-connect-to-local-mysql-server-through-socket-var-run}
  \newline last used: 2014.12.05, 13:15
 
 \bibitem{mysqldumpman} 
  \textbf{Mysql Manual 5.5}, mysqldump — A Database Backup Program\\
  \textit{http://dev.mysql.com/doc/refman/5.5/en/mysqldump.html}
  \newline last used: 2014.12.05, 13:55
    
  \bibitem{mysqlbackupandrectypesman} 
  \textbf{Mysql Manual 5.5} Backup and Recovery Types\\
  \textit{http://dev.mysql.com/doc/refman/5.5/en/backup-types.html}
  \newline last used: 2014.12.06, 19:01
  
  \bibitem{mysqlenterprisebackup} 
  \textbf{Mysql Website}MySQL Enterprise Backup\\
  \textit{http://www.mysql.com/products/enterprise/backup.html}
  \newline last used: 2014.12.08, 10:49

  \bibitem{so1} 
  \textbf{StackOverflow}, RolandoMySQLDBA - answered Mar 7 '12 at 17:45 St\\
  \textit{http://serverfault.com/questions/367255/linux-mysql-is-it-safe-to-copy-mysql-db-files-with-cp-command-from-one-db-to
}
  \newline last used: 2014.12.08, 11:39


 \bibitem{mysqlhotcopyman} 
  \textbf{Mysql Manual 5.5}mysqlhotcopy — A Database Backup Program\\
  \textit{http://dev.mysql.com/doc/refman/5.5/en/mysqlhotcopy.html}
  \newline last used: 2014.12.08, 14:16

 \bibitem{mysqlhtcoptut} 
  \textbf{Backup and Restore MySQL Database using mysqlhotcopy},  Ramesh Natarajan \\
  \textit{http://www.thegeekstuff.com/2008/07/backup-and-restore-mysql-database-using-mysqlhotcopy/}
  \newline last used: 2014.12.08, 14:44
  
  \bibitem{balbalbla} 
  \textbf{How To Backup MySQL Databases on an Ubuntu VPS},  Justin Ellingwood \\
  \textit{https://www.digitalocean.com/community/tutorials/how-to-backup-mysql-databases-on-an-ubuntu-vps}
  \newline last used: 2014.12.08, 15:30
  
  
\end{thebibliography}
\end{document}
